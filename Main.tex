
% Por José Ili\'c García
% Este documento está actualizado a las guías del sistema de bibliotecas de la Universidad de Concepción, solo debes cambiarles los datos correspondientes a tu situación.2

\documentclass[spanish, a4paper, 12pt, twoside, openany]{book} 

% a4papper: Tamaño de papel
% 12pt:     Tamaño de fuente 
% twoside:  Escribir capítulos por ambas caras (Solo sirve si se usa Book, de lo contrario no importa)
% openany:  Inicia capítulos en números pares e impares. 
% book:     Formato libro, está dividido por capítulos. Otros formatos: report, letter, article, etc.

\setlength{\headheight}{15pt} %* si se quita solo provoca un warning

% -------------- Setup, no cambiar ---------------
\usepackage{textcomp}                           % Agrega muchos símbolos extra para utilizar en el texto
\usepackage[T1]{fontenc, url}                   % Para los enlaces url y las fuentes
\usepackage[utf8]{inputenc}                     % Introduce la codificación utf8 que permite el uso de tildes y otros.
\usepackage{titlesec}                           % Suplemento a fancyhdr
\setcounter{secnumdepth}{4}                     % El contador de figuras, secciones, tablas, etc. llega hasta profundidad de 4. i.e. 1.1.1.1
\usepackage{multirow}                           % Extensión para tablas
%\usepackage{minted}                             % Permite darle formato al código que se muestre.
\usepackage{adjustbox}                          % Permite mas versatilidad para cosas como includegraphics.
\usepackage{graphicx}                           % Para poner imágenes
\usepackage{amsmath, amssymb, amsthm}           % paquetes matemáticos
\usepackage{parskip}                            % Elimina la sangría
\urlstyle{sf}                                   % Tipo de url
\usepackage{color}                              % Agrega colores.
\usepackage{subcaption}                         % Permite subtextos en las figuras. Es incompatible con el paquete subfig y subfigure
\usepackage[toc,page]{appendix}                 % Permite el uso de apéndices
\usepackage{chngcntr}                           % Necesario para enumerar las tablas
\counterwithin{table}{section}                  % Numeración de tablas 
\counterwithin{figure}{section}                 % Numeración de figuras
\numberwithin{equation}{section}                % Numeración de ecuaciones
\hyphenpenalty=100000                           % No se dividen las palabras al terminar una línea
\sloppy                                         % Ayuda a que las cosas no se salgan de los márgenes.
\raggedbottom                                   % Hace que el tamaño de las páginas llegue hasta donde llegue el texto, no agrega espacio vertical
\usepackage{xparse,nameref}                     % Ayuda en la creación de nuevos comandos
\usepackage[bottom,hang,flushmargin]{footmisc}  % notas al pie de página están fijas al fondo y sin sangría
\interfootnotelinepenalty=10000                 % Previene que las notas al pie de página se pasen a la siguiente página cuando son muy largas
\usepackage{lipsum}                             % Para crear texto de relleno.

% --------- Editar de aquí en adelante --------
\usepackage{dsfont}
\usepackage{amsfonts}
\usepackage[italic]{hepparticles}
\usepackage{linop}
\usepackage{physics}
\usepackage{tensor}
\newcommand{\im}{\mathsf{i}}
\newcommand{\tl}{\textit{timelike}}
\newcommand{\nl}{\textit{null}}
\usepackage{mathrsfs}


% ----- Apariencia e idioma ----- 
\usepackage[spanish]{babel}                                         % Idioma
\graphicspath{{Images/}{../Images/}}                                % Dónde estarán las imágenes
\usepackage[left=4cm,top=4cm,bottom=2.5cm,right=2.5cm]{geometry}    % Márgenes del documento
\usepackage{setspace}                                               % Permite elegir el interlineado
\linespread{1.3}                                                    % Interlineado de uno y medio. 1.6 es interlineado doble.
\usepackage{microtype}                                              % Permite la modificación de los caracteres.


% ----- Secciones ----- % ESTA PARTE SE UTILIZA EN CASO DE USAR LA CLASE ARTICLE

% \titleformat*{\section}{\LARGE\bfseries}                  % Forma del título de \section 
% \titleformat*{\subsection}{\Large\bfseries}               % Forma del  título de \subsection
% \titleformat*{\subsubsection}{\large\bfseries}            % Forma del  título de \subsubsection 

% Las siguientes tres líneas crean el comando \paragraph con la forma del título correcta.

% \titleformat{\paragraph} 
% {\normalfont\normalsize\bfseries}{\theparagraph}{1em}{}
% \titlespacing*{\paragraph}
% {0pt}{3.25ex plus 1ex minus .2ex}{1.5ex plus .2ex}
%-----------------------------------------------

% ----- Figuras y tablas ----- 
\usepackage{fancyhdr}                           % Permite formatear las cabeceras, pies, enumeración, etc.
\usepackage{subfiles}                           % Para agregar los capítulos que se escriben aparte.
\usepackage{array}                              % Para ordenar texto y ecuaciones.
\usepackage[rightcaption]{sidecap}              % Permite agregar texto lateral
\usepackage{wrapfig}                            % Permite poner figuras con texto al rededor.
\usepackage{float}                              % Permite poner figuras en cualquier lugar.
\usepackage[labelfont=bf]{caption}              % Texto en negrita para descripciones (\caption)
\usepackage[para]{threeparttable}               % Tablas vistosas, mirar antes de utilizar.
\usepackage{url}                                % Permite el uso de enlaces URL.
\usepackage[table,xcdraw,dvipsnames]{xcolor}    % Agranda la cantidad de colores.
\usepackage{makecell}                           % Ayuda en la creación de tablas
\usepackage{hhline}                             % Agranda las opciones de las líneas
\usepackage{textcomp}                           % Símbolo de derechos de autor


% ----- Referencias -----
\usepackage{natbib}                                                     % Ambiente de referencias utilizado.
\bibliographystyle{apalike}                                                 % Estilo de referencias APA.
%\usepackage[nottoc]{tocbibind}
\def\biblio{\clearpage\bibliographystyle{apalike}\bibliography{References}} % Define el comando \biblio para referencias en subarchivos- NO CAMBIAR


% ----- Cabecera y pies -----
\pagestyle{fancy}                           % Se define el estilo fancy
\fancyhead[RO,LE]{\thepage}                 % Número de página en la izquierda para par y derecha para impar
\fancyhead[RE,LO]{\nouppercase{\rightmark}} % Nombre del capítulo en la derecha para par y la izquierda para impar en la cabecera
%\renewcommand{\headrulewidth}{0pt}         % Cambiar para línea más gruesa
\fancyfoot{}                                % Saca el número de la página abajo.

\fancypagestyle{plain}{                     % Se redefine el estilo automático (plain) para que calce con el resto. En particular la 1ra página de cada capítulo
\fancyhf{}                                  % Elimina la cabecera y los pies
\fancyhead[RO,LE]{\thepage}                 % Número de página en la izquierda para par y derecha para impar
\fancyhead[RE,LO]{\nouppercase{\leftmark}}  % Nombre del capítulo en la derecha para par y la izquierda para impar en la cabecera
%\renewcommand{\headrulewidth}{0pt}         % Cambiar para línea más gruesa
\fancyfoot{}                                % Elimina el número de la página abajo
}

%------------------- Cabecera del Resumen y Agradecimientos--------------

\fancypagestyle{resumen}{                   % Se redefine el estilo resumen para que calce con el resto. 
\fancyhf{}                                  % Elimina la cabecera y los pies
\fancyhead[RO,LE]{\thepage}                 % Número de página en la izquierda para par y derecha para impar
\fancyhead[RE,LO]{\nouppercase{Resumen}}    % Nombre del capítulo en la derecha para par y la izquierda para impar en la cabecera
%\renewcommand{\headrulewidth}{0pt}         % Cambiar para línea más gruesa
\fancyfoot{}                                % Elimina el número de la página abajo
}

\fancypagestyle{abstract}{                  % Se redefine el estilo resumen para que calce con el resto. 
\fancyhf{}                                  % Elimina la cabecera y los pies
\fancyhead[RO,LE]{\thepage}                 % Número de página en la izquierda para par y derecha para impar
\fancyhead[RE,LO]{\nouppercase{Abstract}}   % Nombre del capítulo en la derecha para par y la izquierda para impar en la cabecera
%\renewcommand{\headrulewidth}{0pt}         % Cambiar para línea más gruesa
\fancyfoot{}                                % Elimina el número de la página abajo
}

\fancypagestyle{agradecimientos}{                   % Se redefine el estilo resumen para que calce con el resto. 
\fancyhf{}                                          % Elimina la cabecera y los pies
\fancyhead[RO,LE]{\thepage}                         % Número de página en la izquierda para par y derecha para impar
\fancyhead[RE,LO]{\nouppercase{Agradecimientos}}    % Nombre del capítulo en la derecha para par y la izquierda para impar en la cabecera
%\renewcommand{\headrulewidth}{0pt}                 % Cambiar para línea más gruesa
\fancyfoot{}                                        % Elimina el número de la página abajo
}

% ----- Cabecera de la portada ----- 
\fancypagestyle{frontpage}{             % Se define el estilo frontpage.
	\fancyhf{}                          % Elimina la cabecera y los pies
	\renewcommand{\headrulewidth}{0pt}  % Elimina líneas en cabecera
	\renewcommand{\footrulewidth}{0pt}  % Elimina líneas en pies
	\vspace*{1\baselineskip}
	
	\fancyhead[R]{UNIVERSIDAD NACIONAL DE SAN ANTONIO ABAD \linebreak DEL CUSCO                        % Línea 1 de la cabecera derecha, debe ir el nombre de la UdeC en mayúsculas.
	\linebreak       FACULTAD DE CIENCIAS}   % Línea 2 de la cabecera derecha, debe ir el nombre de la facultad en mayúsculas.
	\fancyhead[L]{ \includegraphics[width=0.7in]{./Images/escudo_unsaac.jpg}} % LOGO que irá en la cabecera izquierda.
	\setlength{\headheight}{70pt} %* si se quita solo provoca un warning, pero no un error.
}

% ----- Enlaces clickeables --------
\usepackage{hyperref}   % Permite que todo el documento sea clickeable.
\newcommand\myshade{85} % Permite la redefinición de colores a gusto del usuario

% Para elegir colores propios mirar los nombres relacionados con dvipsnames, aquí un url con los nombres de dvipsnames: https://www.overleaf.com/learn/latex/Using_colours_in_LaTeX

\colorlet{mylinkcolor}{DarkOrchid}   % Define Color de los enlaces clickeables internos (índice, referencias cruzadas, etc.). En este caso DarkOrchid
\colorlet{mycitecolor}{YellowOrange} % Define Color de las citas ubicadas en las referencias. En este caso YellowOrange
\colorlet{myurlcolor}{Aquamarine}    % Define Color de los enlaces url. En este caso Aquamarine

% Para dejar el documento sin texto en colores cambiar las tres líneas anteriores a Black.

\hypersetup{  %Define la forma en que se verán las cosas clickeables.
  	linkcolor  = mylinkcolor!\myshade!black,    % Aplica el color definido arriba. En este caso DarkOrchid
  	citecolor  = mycitecolor!\myshade!black,    % Aplica el color definido arriba. En este caso YellowOrange
  	urlcolor   = myurlcolor!\myshade!black,     % Aplica el color definido arriba. En este caso Aquamarine
  	colorlinks = true,                          % Elimina las cajas al rededor de lo clickeable y lo reemplaza por palabras a color.
}

%--------------------------------------------------------------------------------------------------------------------------
% ------------------------------------------ Aquí empieza el documento ----------------------------------------------------
%--------------------------------------------------------------------------------------------------------------------------


\begin{document}
\def\biblio{}   % Resetea el comando biblio, de lo contrario una lista de referencias será producida después de cada capítulo
% resets the biblio command, if not here a new reference list will be produced after every chapter

\subfile{Otros/Portada}   % Genera la portada desde el archivo Portada.tex . Para editar este archivo ir a Otros -> Portada.tex
\vfill

%----------------Página de derechos de autor: elegir entre a) o b) y borrar/comentar la opción NO utilizada-----------------
\thispagestyle{empty}
\mbox{}                         % Ayuda a bajar el texto
\vfill                          % Deja el texto al fondo
\textcopyright\ 2022, Weyner E. Ccuiro M. \\ % Derechos de autor
%a)
Ninguna parte de esta tesis puede reproducirse o transmitirse bajo ninguna forma o por ningún medio o procedimiento, sin permiso por escrito del autor.\\\\
%b)
Se autoriza la reproducción total o parcial, con fines académicos, por cualquier medio o procedimiento, incluyendo la cita bibliográfica del documento
\vspace{1cm}    % lo separa del fondo
\restoregeometry % Devuelve los márgenes después de la portada


%----------------Página de calificaciones (opcional), descomentar para generar-----------------

% Editar en Otros -> Calificaciones.tex

%\include{Otros/Calificaciones.tex}         % Genera la pagina de calificaciones del archivo calificaciones.tex
%\restoregeometry                           % Devuelve los márgenes después de la página

%\pagenumbering{gobble}         % Suprime la numeración de páginas
%\thispagestyle{plain}          % suprime el encabezado
%\clearpage\mbox{}\clearpage    % Agrega página en blanco

%----------------Página de dedicatoria (opcional), descomentar para generar ---------------------------------

% Editar en Otros -> Dedicatoria.tex

\subfile{Otros/Dedicatoria}   % Genera la pagina de calificaciones del archivo calificaciones.tex
\restoregeometry        % Devuelve los márgenes después de la página


%-----------------Página de agradecimientos (opcional), se incluye normalmente-------------------

% Editar en Otros -> Agradecimientos.

\pagenumbering{roman}                            % Empieza la enumeración romana en minúsculas, para mayúsculas usar Roman.
\newpage
\addcontentsline{toc}{chapter}{AGRADECIMIENTOS}  % Agrega esta sección al índice
\section*{AGRADECIMIENTOS}                       % Debe ir en mayúsculas por reglamento de la UdeC, tiene asterisco para no ser numerada.
\subfile{Otros/Agradecimientos}              % Llama al archivo Agradecimientos en la carpeta Otros.


%-----------------Página de resumen (abstract)-------------

% Si la unidad académica lo requiere, se edita en  Otros -> Resumen.tex . El mismo resumen puede ser incluido en inglés (abstract) en la página siguiente, para agregarlo hay un espacio destinado en el mismo archivo antes mencionado.

\newpage
\addcontentsline{toc}{chapter}{Resumen} % Agrega esta sección al índice
\section*{Resumen (Aún falta agregar)}                      % Con asterisco para que no sea numerada.
\subfile{Otros/Resumen}             % Llama al archivo Resumen en la carpeta Otros.

%--------------Página de índice.  

%\nocite{*}     % Des-comentar si se desea que TODAS las referencias sean impresas en la lista de referencias, incluyendo las que no fueron finalmente citadas en el texto.

\newpage
{\setstretch{1.0}   % Interlineado de la lista.
	\tableofcontents
}

\newpage
{\setstretch{1.0}
	\listoftables}

\newpage
{\setstretch{1.0}
	\listoffigures}


\newpage
\addtocontents{toc}{\protect\setcounter{tocdepth}{4}}   % La profundidad del índice queda en 4, 1.1.1.1
\pagenumbering{arabic}                                  % Comienza la numeración arábiga (números normales)
\setcounter{page}{1}                                    % Comienza el contador de páginas en 1

% A continuación se dejan nombres de diversos capítulos o secciones, para cambiar el nombre del archivo tan solo se debe hacer en la carpeta "capitulos" y luego llamarlos de la forma correcta en "\subfile{Capitulos/nuevonombre}".
% Los nombres de los archivos no pueden llevar tíldes ni espacios para el correcto funcionamiento del compilador, esto no tiene nada que ver con que tengan o no tilde en el documento final.

\chapter{Introducción}                      % Nombre de section/chapter, este si usa tilde y es lo que se verá impreso
\subfile{Capitulos/01Introduccion}      % Incluye el subarchivo del section/chapter 
\clearpage                                  % limpia la página luego de que el capítulo concluya.

\chapter{Marco Teórico}
\subfile{Capitulos/02MarcoTeorico}
\clearpage

\chapter{Metodología}
\subfile{Capitulos/03Metodologia}
\clearpage

\chapter{Análisis}
\subfile{Capitulos/04Analisis}
\clearpage

\chapter{Discusión}
\subfile{Capitulos/05Discusion}
\clearpage

\chapter{Conclusión}
\subfile{Capitulos/06Conclusion}
\clearpage

\newpage

\renewcommand\refname{Referencias}          % Nombre para la lista de referencias, también se utiliza "Bibliografía"
{\setstretch{1.0}                           % Interlineado de las referencias 
	\addcontentsline{toc}{chapter}{Referencias} % Cambia el nombre de la lista de referencias en el índice 
	\bibliography{Referencias}              % Agrega las referencias al documento, estas se ubican en el archivo Referencias.bib
}

\newpage
\renewcommand{\appendixpagename}{Apéndices}     % Nombre al inicio.
\addcontentsline{toc}{chapter}{Apéndices}       % Agrega "Apéndices" al índice

\appendix   % Empieza el ambiente de apéndices, desde ahora en adelante los capítulos, secciones, tablas, figuras, etc. vuelven a empezar su numeración

\chapter{Test}                      % Nombre del apéndice, el capítulo será "Apéndice A" en vez de "Capítulo 1"
\subfile{Capitulos/Apendice}    % Llama al archivo apéndice
\clearpage

% Este segundo apéndice tiene información general sobre LaTeX. Está comentado para que no aparezca en tu documento, pero si lo deseas puedes descomentarlo para ver su contenido.

%\chapter{Algunos consejos sobre \LaTeX{}}
%    \subfile{Otros/Consejos}
%\clearpage

% ÉXITO EN TU TESIS

\end{document}