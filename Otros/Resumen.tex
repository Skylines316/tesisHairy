\documentclass[../Main.tex]{subfiles}
\begin{document}
El estudio de los agujeros negros nos acerca al entendiemiento de la realidad, ya que en estos sistemas podemos observar simultanemente un sistema con gravedad de alta intensidad y alta energía. Esto nos permite estudiar los efectos de la gravedad a nivel cuántico. 

En el presente trabajo yo estudiaré una clase especial de agujeros negros. La característica distintiva es la presencia de un campo escalar fuertemente acoplado. A este tipo de agujero negro le llamaremos agujero negro con pelo o \textit{``hairy black holes''}. Estudiaré las trayectorias que siguen las partículas dentro de un espacio - tiempo deformado por este tipo de agujero negro que llamaremos geodésicas.
%\lipsum[1-2] % Texto para mostrar la página, reemplazar por resumen


\par\vspace*{\fill} % Mueve las palabras clave al final de la página
\textbf{\textit{Keywords --}} Geodésicas, Agujeros negros, Relatividad General %Agregar todas las palabras claves asociados con la tesis aquí.

%-----------Si se desea poner el Abstract Des-comentar lo siguiente-----------
\newpage
\addcontentsline{toc}{chapter}{Abstract} %Agrega esta sección al índice
\section*{Abstract}
The study of black holes brings us closer to the understanding of reality since in these systems we can simultaneously observe a system with high intensity and high energy gravity. This allows us to study the effects of gravity at the quantum level. In the present work, I will study a special class of black holes. The distinguishing feature is the presence of a strongly coupled scalar field. We will call this type of black hole hairy black holes. I will study the trajectories that the particles follow within a space-time deformed by this type of black hole that we will call geodesic.
%\lipsum[1-2] % Texto para mostrar la página, reemplazar por abstract


\par\vspace*{\fill} % Mueve las palabras clave al final de la página
\textbf{\textit{Keywords --}} Geodesics, Black Holes, General Relativity,  % Agregar las palabras claves en inglés

\biblio %Se necesita para referenciar cuando se compilan subarchivos individuales - NO SACAR
\end{document}