% !TeX root = ../Main.tex
\documentclass[../Main.tex]{subfiles}

\begin{document}
Considerando la acción :
\begin{equation}
    S=\int_{\mathcal{M}} \dd[4]{x}\sqrt{-g}R
\end{equation}


Asumiendo un espacio axiosimétrico y estacionario, la métrica más general que se puede escribir es de la siguiente manera(Papapetrou):
%
\begin{equation}
    \dd{s}^{2} = -f(\rho,z)\qty(\dd{t}-\omega\dd{\phi})^{2}+f^{-1}(\rho,z)\qty[\rho^{2}\dd{\phi}^{2}+e^{2\gamma(\rho,z)}\qty(\dd{\rho}^{2}+\dd{z}^{2})]
    \label{papapetrou}
\end{equation}
%
luego asumamos para mayor simplicidad que $\omega=0$, entonces la métrica es:
%
\begin{equation}
    \dd{s}^{2} = -f(\rho,z)\dd{t}^{2}+f^{-1}(\rho,z)\qty[\rho^{2}\dd{\phi}^{2}+e^{2\gamma(\rho,z)}\qty(\dd{\rho}^{2}+\dd{z}^{2})]
    \label{metric}
\end{equation}
%
y al exigir que $R_{\mu\nu}=0$, además de escoger convenientemente, primero $R_{zz}-R_{\rho\rho}=0$:
\begin{equation}
    \pdv{\gamma}{\rho}=\dfrac{1}{4}\dfrac{1}{f^{2}(\rho,z)}\qty[\qty(\pdv{f(\rho,z)}{\rho})^{2}-\qty(\pdv{f(\rho,z)}{z})^{2}]
\end{equation}
%
y luego de $R_{\phi\phi}=0=R_{tt}$:
\begin{equation}
    f(\rho,z)\qty(\dfrac{1}{\rho}\pdv{f(\rho,z)}{\rho}+\pdv[2]{f(\rho,z)}{\rho}+\pdv[2]{f(\rho,z)}{z})=\qty[\qty(\pdv{f(\rho,z)}{\rho})^{2}+\qty(\pdv{f(\rho,z)}{z})^{2}]
\end{equation}
%
Lo cuál se puede escribir como:
\begin{equation}
    f(\rho,z)\laplacian{f(\rho,z)}=\grad{f(\rho,z)}\vdot\grad{f(\rho,z)}
\end{equation}
%
tengamos en cuenta que los operadores $\laplacian$ y $\grad$ son operadores definidos en un espacio euclideo en coordenadas cilíndricas con la métrica $\dd{s}^{2} = \rho^{2}\dd{\phi}^{2}+\dd{\rho}^{2}+\dd{z}^{2}$.
%
\newline
%
y finalmente de $R_{\rho z}=0$:
\begin{equation}
    \pdv{\gamma}{z}=\dfrac{1}{2}\rho\dfrac{1}{f^{2}(\rho,z)}\qty(\pdv{f(\rho,z)}{\rho}\pdv{f(\rho,z)}{z})
\end{equation}
%
ahora juntando todas estas condiciones para las funciones tenemos que:
\begin{eqnarray}
    \label{Function_f}
    f(\rho,z)\laplacian{f(\rho,z)}= & \grad{f(\rho,z)}\vdot\grad{f(\rho,z)}\\
    \label{gamma_partial_rho}
    \pdv{\gamma}{\rho}= & \dfrac{1}{4}\dfrac{1}{f^{2}(\rho,z)}\qty[\qty(\pdv{f(\rho,z)}{\rho})^{2}-\qty(\pdv{f(\rho,z)}{z})^{2}] \\
    \label{gamma_partial_z}
    \pdv{\gamma}{z}= & \dfrac{1}{2}\rho\dfrac{1}{f^{2}(\rho,z)}\qty(\pdv{f(\rho,z)}{\rho}\pdv{f(\rho,z)}{z})
\end{eqnarray}
% \begin{eqnarray}
%     \label{firstcon}
%     \grad{\qty(f^{-1}\grad{f}+\rho^{-2}f^{2}\omega\grad{\omega})} & =0 \\
%     \label{secondcon}
%     \grad{\qty(\rho^{-2}f^{2}\grad{\omega})} & =0 \\
%     \label{thirdcon}
%     \dfrac{1}{4}\rho f^{-2}\qty[\qty(\pdv{f}{\rho})^{2}-\qty(\pdv{f}{z})^{2}]-\dfrac{1}{4}\rho^{-1}f^{2}\qty[\qty(\pdv{\omega}{\rho})^{2}-\qty(\pdv{\omega}{z})^{2}] & =\pdv{\gamma}{\rho} \\
%     \dfrac{1}{2}\rho f^{-2}\pdv{f}{\rho}\pdv{f}{z}-\dfrac{1}{2}\rho^{-1} f^{2}\pdv{\omega}{\rho}\pdv{\omega}{z} & = \pdv{\gamma}{z}
%     \label{fourthcon}
% \end{eqnarray}
%
% desarrollando \eqref{firstcon} con ayuda de \eqref{secondcon} tenemos:
%
% \begin{equation}
%     f\laplacian{f}=\grad{f}\vdot\grad{f}-\rho^{-2}f^{4}\grad{\omega}\vdot\grad{\omega}
%     \label{conduno}
% \end{equation}
%
% Ahora definamos una función $\varphi$ que no depende del azimut, entonces esta función cumple la siguiente identidad:

% \begin{equation}
%     \grad(\rho^{-1}\vu{n}\cp\grad{\varphi})=0
% \end{equation}

% donde $\vu{n}$ es un vector unitario en la dirección del azimut. Ahora de \eqref{secondcon} podemos decir que es la condición de integrabilidad de un función $\varphi$ definida por:

% \begin{equation}
%     \rho^{-1}f^{2}\grad{\omega}=\vu{n}\cp\grad{\varphi}
% \end{equation}

% esta relación es equivalente a:

% \begin{equation}
%     f^{-2}\grad{\varphi}=-\rho^{-1}\vu{n}\cp\grad{\omega}
% \end{equation}

% esto implica que

% \begin{equation}
%     \grad(f^{-2}\grad(\varphi))=0
% \end{equation}

% si expresamos \eqref{conduno} en función de este nuevo potencial $\varphi$. Podemos definir la función $\mathcal{E}$.

% \begin{equation}
%     \mathcal{E}=f+\im \varphi
% \end{equation}
%
si fijamos que la parte real de una cierta función $\mathcal{E}$ es igual a $f(\rho,z)$, entonces la ecuación \eqref{Function_f} se reescribe como:
%
\begin{equation}
    \qty(\Re{\mathcal{E}})\laplacian{\mathcal{E}}=\grad{\mathcal{E}}\vdot\grad{\mathcal{E}}
    \label{princCondUno}
\end{equation}
%
uno puede hacer simples modificaciones a $\mathcal{E}$. Una de las importantes es la siguiente(Papapetrou):
%
\begin{equation}
    \mathcal{E}=\qty(\xi-1)/\qty(\xi+1)
    \label{Eecu}
\end{equation}
%
cuando reemplazemos $\mathcal{E}=\mathcal{E}(\xi)$ en \eqref{princCondUno} obtenemos la siguiente ecuación para $\xi$:
%
\begin{equation}
    \qty(\xi\xi^{*}-1)\laplacian{\xi}=2\xi^{*}\grad{\xi}\vdot\grad{\xi}
    \label{xiEcu}
\end{equation}
%
esta ecuación tambien se puede derivar de principio variacional de:
%
\begin{equation}
    \delta\int\dfrac{\grad{\xi}\vdot\grad{\xi}^{*}}{\qty(\xi\xi^{*}-1)^{2}}\dd{v}=0
\end{equation}
%
De \eqref{xiEcu} podemos notar que si $\xi$ es una solución entonces $e^{\im\alpha}\xi$ también es una solución. Entonces podemos expresar la función $\xi$ de la siguiente manera:
%
\begin{equation}
    \xi=-e^{\im\alpha}\coth{\psi}
    \label{psiEcu}
\end{equation}
%
donde al reemplazar esta expresión para $\xi$ en \eqref{xiEcu} se tiene que $\psi$ cumple la ecuación de laplace
%
\begin{equation}
    \laplacian{\psi}=0
\end{equation}
%
Por consiguiente se puede expresar $\psi$ en terminos de una ``expansión multipolar''. Las soluciones ya fueron estudiadas por Weyl en 1917 cuando $\alpha=0$ y por Papapetrou en 1953 cuando $\alpha=\pi/2$. Desafortunadamente cuando $\alpha \neq 0$ (mod $\pi$) uno puede excluir la contribución monopolar de $\psi$ si el espacio es asintóticamente plano. Por lo que esas soluciones no poseen un gran valor físico.
%
Ahora es mejor expresar la solucion en ``prolate spheroidal coordinates'' (zipoy). Entonces fijamos
%
\begin{eqnarray}
    \rho= & (x^2-1)^{1/2}(1-y^{2})^{1/2} \\
    z = & xy
\end{eqnarray}
%
cuando realizamos este cambio de coordenadas $\psi$ puede ser expresado como una superposisión lineal de funciones de Legendre:
%
\begin{equation}
    \psi=\sum_{l}\alpha_{l}Q_{l}(x)P_{l}(y)
\end{equation}
%
donde $Q_{l}(x)$ son las funciones de Legendre de segunda clase y $P_{l}(y)$ son las funciones de Legendre de primera clase. En el caso puro de $l=0$ tenemos
%
\begin{equation}
    \psi=\frac{1}{2}\ln(\dfrac{x+1}{x-1})
\end{equation}

lo cual reemplazando en \eqref{psiEcu} tenemos la solución para $\xi$:

\begin{equation}
    \xi = x
\end{equation}

Esta solución corresponde a Schwarzschild. Donde $x+1$ es la coordenada radial en Schwarzschild (escogiendo $M$ como unidad de longitud) y $y=\cos(\theta)$. Es decir en coordenadas clásicas ($c=1$) tenemos la siguiente transformación de coordenadas:

\begin{eqnarray}
    \label{xTor}
    x &= \dfrac{r}{M}-1 \\
    \label{yToTheta}
    y &= \cos(\theta)
\end{eqnarray}

entonces reemplazando en \eqref{Eecu} tenemos

\begin{equation}
    \mathcal{E}=\qty(x-1)/\qty(x+1)
\end{equation}

donde $x$ es una variable real por lo que podemos identificar que

\begin{eqnarray}
    \label{fecures}
    f = & \qty(x-1)/\qty(x+1) \\
    \label{varphiecures}
    \varphi = & 0
\end{eqnarray}

entonces en coordenadas clásicas tenemos que

\begin{eqnarray}
    \label{fecuresSch}
    f = & 1-\dfrac{2M}{r} \\
    \label{omegaecuresSch}
    \omega = & 0
\end{eqnarray}

Des esta manera reemplzando en \eqref{papapetrou} tenemos que

\begin{equation}
    \dd{s}^{2} = -\qty(1-\dfrac{2M}{r})\dd{t}^{2}+\qty(1-\dfrac{2M}{r})^{-1}\qty[\rho^{2}\dd{\phi}^{2}+e^{2\gamma}\qty(\dd{\rho}^{2}+\dd{z}^{2})]
\end{equation}

donde

\begin{eqnarray}
    \label{rhoTortheta}
    \rho = & r\qty(1-\dfrac{2M}{r})^{1/2}\sin (\theta) \\
    \label{zTortheta}
    z = & \qty(r-M)\cos(\theta) \\
    \label{r}
\end{eqnarray}

Entonces

\begin{multline*}
    \dd{s}^{2} = -\qty(1-\dfrac{2M}{r})\dd{t}^{2}+r^{2}\sin(\theta)^{2}\dd{\phi}^{2}\\
    +\qty(1-\dfrac{2M}{r})^{-1}e^{2\gamma}\qty(\dfrac{r^{2}-2mr+m^{2}\sin(\theta)^{2}}{r^{2}})\qty[\qty(1-\dfrac{2M}{r})^{-1}\dd{r}^{2}+r^{2}\dd{\theta}^{2}]
\end{multline*}

\textcolor{blue}{Asumiendo que}

\begin{equation*}
    \qty(1-\dfrac{2M}{r})^{-1}e^{2\gamma}\qty(\dfrac{r^{2}-2mr+m^{2}\sin(\theta)^{2}}{r^{2}})=1
\end{equation*}

tenemos la clásica métrica de Schwarzschild.

\begin{equation}
    \dd{s}^{2} = -\qty(1-\dfrac{2M}{r})\dd{t}^{2}+r^{2}\sin(\theta)^{2}\dd{\phi}^{2}+\qty(1-\dfrac{2M}{r})^{-1}\dd{r}^{2}+r^{2}\dd{\theta}^{2}
\end{equation}

\subsection{Reissner-Nordström}
El análisis se realiza de manera similar solo que en este caso en el lagrangiano se agrega el campo electromagnético. Es por eso que la función $\mathcal{E}$ ahora tiene la forma

\begin{equation}
    \mathcal{E}=\qty(f-\abs{\Phi}^{2})+\im\varphi
\end{equation}

y cuando las nuevas restricciones de este caso se aplican tenemos que como conocemos

\begin{equation}
    \mathcal{E}=\qty(\xi-1)/\qty(\xi+1)
\end{equation}

y con respecto al neuvo campo

\begin{equation}
    \Phi=q/(\xi+1)
\end{equation}

se utilizan las mismas coordenadas $x,y$. Entonces usando nuestra solución anterior para $\xi$ tenemos

\begin{equation}
    \mathcal{E}=\qty(x-1)/\qty(x+1)
\end{equation}

de donde podemos despejar $f$

\begin{equation}
    f=\mathcal{E}+\abs{\Phi}^{2}
\end{equation}

reemplazando obtenemos

\begin{equation}
    f=\dfrac{\qty(x-1)}{\qty(x+1)}+\dfrac{q^{2}}{\qty(x+1)^{2}}
\end{equation}

con la transformación de coordenadas que definimos y teniendo en cuenta que $Q=qm$ tenemos

\begin{equation}
    f=1-\dfrac{2M}{r}+\dfrac{Q^{2}}{r^{2}}
\end{equation}

por lo que la métrica que obtendremos es

\begin{equation}
    \dd{s}^{2} = -\qty(1-\dfrac{2M}{r}+\dfrac{Q^{2}}{r^{2}})\dd{t}^{2}+r^{2}\sin(\theta)^{2}\dd{\phi}^{2}+\qty(1-\dfrac{2M}{r}+\dfrac{Q^{2}}{r^{2}})^{-1}\dd{r}^{2}+r^{2}\dd{\theta}^{2}
\end{equation}
%\subsection{Este es un sub título}

%\lipsum[15-17] % dummy text

%\subsubsection{Este es un sub sub título}

%\lipsum[15-17] % dummy text

%\paragraph{Este es un párrafo}

%\lipsum[18-19] % dummy text

\biblio %Se necesita para referenciar cuando se compilan subarchivos individuales - NO SACAR
\end{document}