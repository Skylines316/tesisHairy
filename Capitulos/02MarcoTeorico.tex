% !TeX root = ../Main.tex
\documentclass[../Main.tex]{subfiles}

\begin{document}
\section{Relatividad General}
El 5 de Julio de 1687 Isaac Newton publica \textit{Philosophiæ Naturalis Principia Mathematica} \cite{newton87a} en donde publica sus leyes del movimiento y su ley de gravitacion universal. Esta ley de gravitación fue la primera descripción matemática a partir de la cuál podiamos tener ecuaciones de movimiento para predecir el movimiento de los planetas. Sin embargo, esta descripción de la naturaleza aún estaba incompleta. Una de las limitantes de la descripcion de las interacciones segun Newton es la velocidad de interaccion infinita. Esto se puede ver claramente en al expresion matemática de la fuerza de interaccion gravitacional, esta solo depende de las masas y la distancia entre ellas. 

\begin{equation}
    \vec{F} = \frac{Gm_1m_2}{r^2} \vu{r}
\end{equation}

Esta velociad de interacción infinita romperia la causalidad ya que se propagaria más rápido que la luz. Es así que en 1905 Albert Einstein propone su teoría de la relatividad especial \cite{einstein}. En esta nueva teoría se abandona la idea de tiempo absoluto y se adopta la de tiempo propio, es decir que cada observador mide el tiempo de manera difernte, tiene su propio tiempo. Además de que el tiempo y el espacio ya no son entes separados por lo que se les llama en conjunto espaciotiempo.

Desafortunadamente cuando la interacción gravitaria se vuelve importante la relatividad especial falla al tratar de describir la realidad \cite{1994bhtw.bookT}. Es entonces que para poder describir la gravedad, con las caracteristicas ya agregadas por la relatividad especial, se necesita de un nuevo conjunto de leyes físicas que llamamos Relatividad General. En esta descripcion el espaciotiempo se curva y esta curvatura es lo que nosotros interpretamos como la interacción gravitacional. Poniendo el ejemplo de un ser bidimensional que vive en la superficie de uan esfera y coordina con otro para ubicarse en el ecuador y caminar en linea recta, siempre manteniendo un angulo de 90 grados sexagesimales entre sus trayectorias, estos notaran que a medida que caminan se van acercando uno al otro. Entonces estos seres bidimensionales interpretarian esto como una fuerza que los atrae, cuando mas bien es la geometria en donde viven que les hace seguir esa trayectoria.

Para poder describir todos estos conceptos en espacios curvos necesitamos herramientas matemáticas tales como calculo tensorial y geometría diferencial.

\subsection{Variedades}\label{subsec:Variedades}
Cuando el ser bidimensional que poniamos en un ejemplo anterior quiere poder mapear toda su superficie este podria pensar en caracterizarlo con 2 numeros. Sin embargo al hacer esto se perderian caracteristicas globales y por lo tanto estaria cometiendo una grave equivocacion. Es por eso que necesitamos de un objeto matemático que localmente luzca como una espacio $\mathbb{R}^n$ pero que globalmente tenga unas propiedades ligeramente distintas.

\begin{itemize}
    \item [Variedad]
    Una variedad real, $C^{\infty}$, de n dimensiones $\mathcal{M}$. Es un conjunto junto con una coleccion de subconjuntos $\{O_{\alpha}\}$ que satisfacen las siguientes propiedades.
    \begin{itemize}
        \item Cada punto $p \in \mathcal{M}$ esta dentro de almenos un $O_{\alpha}$, es decir que $\{O_{\alpha}\}$ recubre todo $\mathcal{M}$.
        \item Para cada $\alpha$, existe mapa uno a uno $\psi_{\alpha}:O_{\alpha}\rightarrow U_{\alpha}$, donde $U_{\alpha}$ es un subconjunto abierto de $\mathbb{R}^{n}$.
        \item Si dos conjuntos $O_{\alpha}$ y $O_{\beta}$  se sobreponen, $O_{\alpha} \cap O_{\beta} \neq \emptyset$, podemos considerar el mapa $\psi_{\beta} \circ \psi_{\alpha}^{-1}$ la cuál toma puntos en $\psi_{\alpha}[O_{\alpha} \cap O_{\beta}]\subset U_{\alpha}\subset \mathbb{R}^{n}$ a puntos en $\psi_{\beta}[O_{\alpha} \cap O_{\beta}]\subset U_{\beta}\subset \mathbb{R}^{n}$ \footnote{Cada mapa $\psi_{\alpha}$ es generalemente llamado un sistema de coordenadas porque pasas de un palno tangente a otro plano tangente}.
    \end{itemize}

    \item [Vector]
    Un vector es aquel elemento que cumple las propieades de un espacio vectorial. Los espacios vectoriales que veremos son espacios tangentes a un punto $p$ y cumplen que: Sea $\mathcal{M}$ una variedad n-dimensional, sea $p \in \mathcal{M}$ y denotemos por $V_{p}$ al espacio tangente en p. Entonces $\dim{V_{p}} = n$.

    \item [Tensor]
    Un tensor es un mapa multilineal (lineal en cada variable) de vectores (o vectores duales) en números. Es decir que mapeamos todos los vectores en numeros escalares ordenados (este orden puede ser matricialmente).O de manera mas formal, sea V un espacio vectorial finito dimensional y sea $V^{*}$ el espacio vectorial dual\footnote{Este espacio vectorial dual $V^{*}$ tiene la característica que sea $v_{1}, \dots, v_{n}$ la base de V, y sea $v^{1^{*}}, \dots, v^{n^{*}}$ la base de $V^{*}$. Entonces $v^{\mu^{*}}(v_{\nu})=\delta^{\mu}_{\nu}$, donde $\delta^{\mu}_{\nu}=1$ si $\mu=\nu$ y 0 en los otros casos.} de V. Un tensor, T, de tipo (k,l) sobre V es un mapa multilineal.
    \begin{equation}
        T: \underbrace{V^{*}\times\dots\times V^{*}}_\text{k veces} \times \underbrace{V \times \dots \times V}_\text{l veces} \rightarrow \mathbb{R}
    \end{equation}
    \item [Métrica]
    Intuitivamente una métrica nos indica la distancia al cuadrado infinesimal asociado con un desplazamiento infinitesimal. Los desplazamientos infinitesimales estan asociados con los vectores del plano tangente por lo que la métrica debe ser un mapa lineal de $V_{p}\times V_{p}$ en números, esto es un tensor de tipo (0, 2). Las propieades que tiene este tensor son:
    \begin{itemize}
        \item [Simetrico]
        La distancia debe ser simétrica, esto es $g(\vec{v}_{1}, \vec{v}_{2})=g(\vec{v}_{2}, \vec{v}_{1}); \vec{v}_{1}, \vec{v}_{2} \in V_{p}$.
        \item [No degenerada]
        El valor de la métrica debe ser igual para todo observador, esto es $g(\vec{v},\vec{v}_{1})=0 \Leftrightarrow \vec{v}_{1}=0; \vec{v}_{1}, \vec{v} \in V_{p}$
    \end{itemize}
    En otras palabras la métrica es un producto interno del espacio tangente en cada punto, y no esta definida positiva necesariamente. En una base coordenada podemos expandir la métrica en términos de sus componentes $g_{\mu\nu}$ (ver \ref{sec:tensores}).
    \begin{equation}
        ds^{2} = g_{\mu\nu} \dd{x}^{\mu}\dd{x}^{\nu}
    \end{equation}
    %Una variedad es un conjunto de piezas que \textit{lucen} como bolas abiertas en $R^n$\footnote{Generalización de un intervalo abierto n dimensional} que se \textit{cosen} unas a otras suavemente. 
\end{itemize}

En la escritura de las ecuacione se ha seguido una reglas de notación, eso se ve a mayor detalle en \ref{sec:tensores}.

\subsection{Curvatura}
Como pudimos ver en \ref{subsec:Variedades} los vectores estan definidos en los espacios tangentes y contangentes a la variedad. Entonces para poder relacionar vectores de diferentes espacios tanges necesitamos de una conexión. En relatividad general se usa la conexión de Levi-Civita. Esta conexión es especial ya que el espaciotiempo no tendra torsión y solo presentara curvatura.

La derivada de un vector, en coordenas cuyos vectores base son constantes, es otro vector. Y el mapeo de las derivadas de sus componentes conforman un tensor. Esto no sucede así para coordenadas curvilineas o espacios curvos, en este caso necesitamos de un factor extra que compense la curvatura. Este factor extra son los simbolos de Christoffel, así entonces tenemos

\begin{equation}
    \nabla_{\nu}{A}^{\mu} = \partial_{\nu}{A}^{\mu}+\Gamma^{\mu}_{\nu\rho}A^{\rho}
\end{equation}

a esta derivada se le conoce como derivada covariante. Tambien por la conexión de Levi-Civita los símbolso de Christoffel cumple con lo siguiente.

\begin{equation}
    \Gamma^{\mu}_{\nu\rho} = \Gamma^{\mu}_{\rho\nu}
\end{equation}

Y la dervida covariante cumple con

\begin{equation}
    \nabla_{\nu}{g_{\mu\nu}} = 0
\end{equation}

de esta expresión se puede deducir que

\begin{equation}
    \Gamma^{\mu}_{\nu\rho} = \dfrac{1}{2}g^{\mu\sigma}\qty(\partial_{\nu}{g_{\rho\sigma}}+\partial_{\rho}{g_{\nu\sigma}}-\partial_{\sigma}{g_{\nu\rho}})
\end{equation}

A partir del operador de derivada nosotros podemos introducir la noción de transporte paralelo de un vector sobre la curva $C$. Para que este transporte parelelo se dé, la variación de este vector debe ser paralela al vector tangente de la curva $C$, esto es:

\begin{equation}
    t^{a}\nabla_{a}v^{b} = 0 
\end{equation}

Ahora el espacio en el que trabajamos normalmente en realtividad general esta libre de torsióm, esto quiere decir que si una funcion escalar $f$ la variamos en cierto orden $\nabla_{a}\nabla_{b}f$ esta será igual al hacerlo en orden invertido, esto es:

\begin{equation}
    \nabla_{a}\nabla_{b}f = \nabla_{b}\nabla_{a}f
\end{equation}

sin embargo si nosotros hacemos esto para el dual de un campo vectorial esto no sucede, por lo que tenemos que $(\nabla_{a}\nabla_{b}-\nabla_{a}\nabla_{b})\omega_{c}$ no es igual a cero y lo que sucede es que a $(\nabla_{a}\nabla_{b}-\nabla_{b}\nabla_{a})$ define un mapeo lineal de vectores duales $\omega_{c}$ de tipo (0,1) a un tensor de tipo (0,3). Esto es quivalente a la acción de un tensor (1,3) sobre el tensor (0,1). Este tensor que cumple estas características, es el tensor de \textit{curvatura de Riemann} $\tensor{R}{_{abc}^{d}}$. Esto es:

\begin{equation}
    (\nabla_{a}\nabla_{b}-\nabla_{b}\nabla_{a})\omega_{c} = \tensor{R}{_{abc}^{d}}\omega_{d}
\end{equation}

Ahora veamos las propiedades del tensor de Riemann:

\begin{enumerate}
    \item Antisimetría en los 2 primeros índices:
    \begin{equation}
        \tensor{R}{_{abc}^{d}} = -\tensor{R}{_{bac}^{d}}
        \label{Riemann:1}
    \end{equation}

    \item Propiedad equivalente en dormas diferenciales a $\dd[2]{\vec{\omega}}$:
    \begin{equation}
        \tensor{R}{_{[abc]}^{d}} = 0
        \label{Riemann:2}
    \end{equation}

    \item Antisimetría de los 2 últimos índices:
    \begin{equation}
        \tensor{R}{_{abcd}} = -\tensor{R}{_{abdc}}
        \label{Riemann:3}
    \end{equation}

    \item Identidad de Bianchi:
    \begin{equation}
        \nabla_{[a}\tensor{R}{_{bc]d}^e} = 0
        \label{Riemann:4}
    \end{equation}

    \item De 1 a 3 se puede deducir
    \begin{equation}
        \tensor{R}{_{abcd}} = \tensor{R}{_{cdab}}
        \label{Riemann:5}
    \end{equation}
\end{enumerate}

Es útil descomponer al tensor de Riemann en la parte con traza y la parte libre de traza. Por \eqref{Riemann:1} y \eqref{Riemann:3} la traza sobre los 2 primeros o 2 últimos índices es cero. Sin embargo, la traza sobre el segundo y cuarto índice que es equivalente a la traza sobre el primer y tercer índice (esto también por \eqref{Riemann:1} y \eqref{Riemann:3}) define el tensor de \textit{Ricci}, $\tensor{R}{_{ac}}$.

\begin{equation}
    \tensor{R}{_{ac}} = \tensor{R}{_{abc}^{b}}
\end{equation}

de \eqref{Riemann:5} encontramos la siguiente propiedad para el tensor de Ricci

\begin{equation}
    \tensor{R}{_{ac}} = \tensor{R}{_{ca}}
\end{equation}

Finalmente podemos definir la traza del tensor de Ricci como la \textit{Curvatura Escalar}.

\begin{equation}
    R = \tensor{R}{_{a}^{a}}
\end{equation}

y la parte libre de traza es llamada \textit{tensor de Weyl}.

\subsection{Geodésicas}
Usando la descripción Newtoniana de la gravedad se pueden describir las trayectorias de los cuerpos dentro de un campo gravitacional. Estas trayectorias son cónicas, dependiendo de la energía de la partícula serán elípticas, circulares o parabólicas \cite{Landau1976Mechanics}.

En relatividad general a estas trayectorias se les llama geodesicas. Existen 2 tipos de geodesicas, las métricas y las afines. Sin embargo, en la teoría que estudiare ambas son indiferentes. Por lo que de manera general podemos definir a las geodesicas de la siguiente manera.
\begin{itemize}
    \item[Geodesicas:]   
Mi trabajo consiste en encontrar estas trayectorias para un cuerpo dentro de un espacio-tiempo
\end{itemize}

\subsection{Solución de Schwarzschild}
\subsection{Potencial efectivo de Schwarzschild}
\subsection{Geodésicas de Schwarzschild}
\subsubsection{Problema de valores iniciales}
\section{Agujeros Negros con Pelo}
\subsection{Solución con $\Lambda=J=Q=0$}
\subsection{Potencial efectivo de Agujeros Negros con Pelo}
\subsection{Ecuación diferencial de las geodésicas}
\subsection{Geodésicas de Agujeros Negros con Pelo}

%\subsection{Este es un sub título}

%\lipsum[15-17] % dummy text

%\subsubsection{Este es un sub sub título}

%\lipsum[15-17] % dummy text

%\paragraph{Este es un párrafo}

%\lipsum[18-19] % dummy text

\biblio %Se necesita para referenciar cuando se compilan subarchivos individuales - NO SACAR
\end{document}