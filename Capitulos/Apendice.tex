% !TeX root = ../Main.tex
\documentclass[../Main.tex]{subfiles}
\begin{document}

% Las siguientes líneas están para que la numeración en el apéndice sea correcta - NO CAMBIAR.
% Usar estas en caso de utilizar la clase book, de lo contrario comentar
\renewcommand{\thesection}{A\arabic{section}}
\renewcommand{\thetable}{A\arabic{section}.\arabic{table}}
\counterwithin{table}{section}
\counterwithin{figure}{section}

% Usar estas en caso de utilizar cualquier clase que NO SEA book.
%\renewcommand{\thesubsection}{A\arabic{subsection}}
%\renewcommand{\thetable}{A\arabic{subsection}.\arabic{table}}
%\counterwithin{table}{subsection}
%\counterwithin{figure}{subsection}

%---------- Escribir desde aquí en adelante


\section{Simetría axial y estacionaria}

Asumamos un espacio axiosimétrico y estacionario. En este espacio la métrica se puede expresar de la siguiente manera:
%
\begin{equation}
\dd{s}^{2} = -V(\rho, z)\qty(\dd{t}-\omega\dd{\phi})^{2}+V(\rho, z)^{-1}\rho^{2}\dd{\phi}^{2}+\Omega(\rho, z)^{2}\qty(\dd{\rho}^{2}+\Lambda(\rho, z)\dd{z}^{2})
\end{equation}
%
Donde $V,\Omega, \Lambda$ son funciones que solo dependen de $\rho$ y $z$la cual en el caso especial en que el espacio es vacío, es decir $R_{ab}=0$ tenemos la siguiente ecuación para $\rho$:
%
\begin{equation}
    D^{a}D_{a}\rho=0 \quad D_{a}: \text{derivada covariantes en la 2D atravesada por } \rho \, z
\end{equation}
%
% \section{Logo Universidad de Concepción}
% \begin{figure}[h]
%     \centering
% \caption{UdeC logo}
%     \includegraphics[width=0.2\columnwidth]{./Images/escudo_unsaac.jpg}
%     \label{fig_logo2}
% \end{figure}

\section{Tensores covariante y contravariante, métrica}\label{sec:tensores}
Un tensor contravariante es del tipo (k,0) y un tensor covariante es del tipo (0,l). Comunmente nos referiremos a los vectores que estan en el espacio tangente $V_{p}$ como contravariantes y a los vectores que estan en el espacio cotangente $V^{*}_{p}$ como covariantes. Es así que, sea el vector $\vec{v}$ podemos descomponerlo en los vectores base de $V_{p}$\footnote{Tambien se puede descomponer en los vectores base de $V^{*}_{p}$.},y $\dim{v_{p}}=3$ \footnote{Para el ejemplo trabajamos en 3 dimensiones, pero la generalización simple.}.

\begin{equation}
    \vec{v}=v^{1}\vu{e}_1+v^{2}\vu{e}_2+v^{3}\vu{e}_3 = v^{i}\vu{e}_i
\end{equation}

y a partir de la proyeccion de este vector hallaremos las componentes covariantes.

\begin{equation}
    v_{j}=\vec{v}\cdot\vu{e}_{j}=v^{i}\vu{e}_i\cdot\vu{e}_{j}
\end{equation}

es así que podemos definir los componentes de la métrica.

\begin{eqnarray}
    g_{ij} = \vu{e}_i\cdot\vu{e}_{j} & \qq{donde} \vu{e}_i,\vu{e}_j \qq{son vectores base de} V_{p}\\
    g^{ij} = \vu{e}^i\cdot\vu{e}^{j} & \qq{donde} \vu{e}^i,\vu{e}^j \qq{son vectores base de} V^{*}_{p}
\end{eqnarray}

Con esta definición de la métrica tenemos algunas propiedades interesantes

\begin{itemize}
    \item Subir y bajar indices
    \begin{eqnarray}
        v_{j} & = v^{i} g_{ij} \\
        v^{j} & = v_{i} g^{ij}
    \end{eqnarray}
    \item Norma de un vector
    \begin{eqnarray}
        \vec{v}\vdot\vec{v}= & v^{i}\vu{e}_i\vdot v^{j}\vu{e}_j=v^{i}v^{j}\vu{e}_i\vdot\vu{e}_{j}=v^{i}v^{j}g_{ij} \\ 
        & \vec{v}\vdot\vec{v} = v^{i} v_{i}
    \end{eqnarray}
\end{itemize}


\biblio %Se necesita para referenciar cuando se compilan subarchivos individuales - NO SACAR
\end{document}