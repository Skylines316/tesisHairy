% !TeX root = ../Main.tex
\documentclass[../Main.tex]{subfiles}
\begin{document}

% Las siguientes líneas están para que la numeración en el apéndice sea correcta - NO CAMBIAR.
% Usar estas en caso de utilizar la clase book, de lo contrario comentar
\renewcommand{\thesection}{A\arabic{section}}
\renewcommand{\thetable}{A\arabic{section}.\arabic{table}}
\counterwithin{table}{section}
\counterwithin{figure}{section}

% Usar estas en caso de utilizar cualquier clase que NO SEA book.
%\renewcommand{\thesubsection}{A\arabic{subsection}}
%\renewcommand{\thetable}{A\arabic{subsection}.\arabic{table}}
%\counterwithin{table}{subsection}
%\counterwithin{figure}{subsection}

%---------- Escribir desde aquí en adelante
\section{Notacion, convencion de suma de Einstein}
Cuando uno realiza operaciones con tensores, incluso las mas simples manipulaciones pueden resultar en notaciones muy complicadas. Es por eso que necesitamos nuevas convenciones para solucionar estos problemas.
\begin{itemize}
    \item Usaremos la notacion de índices abstractos, segun esta notacion solo usaremos las componentes. A los tensores con componentes contravariantes (k,0) los denotaremos con idices superiores $T^{a_{1}\dots a_{k}}$ y a los ensores con componentes covariantes (0,l) con indices inferiores $T_{b_{1}\dots b_{l}}$. Así los tensores (k,l) serán denotados con indices superiores y inferiores respectivamente, es decir $T^{a_{1}\dots a_{k}}_{b_{1}\dots b_{l}}$, tambien en la multiplicación de tensores obviaremos el simbolo de multiplicación externa $\otimes$ y no repetiremos la letra que indica el índice; esto es
    \begin{equation}
        T^{ijk}_{lm}\otimes S^{i}_{j} \xrightarrow{\text{se convierte en}} T^{ijk}_{lm}S^{n}_{o}
    \end{equation}
    \item Los tensores tienen muchos indices (correspondientes a vectores y vectores duales) y al contraer estos indices aparecen simbolos de suma. Cuando las contracciones son de varios indices aumentan los simbolos de suma, por eso usaremos la convencion de suma de Einstein.
    \begin{eqnarray}
        A^{i}B_{i} & = \sum_{i}A^{i}B_{i} \\
        C^{abcdef}D_{abcijkl} & = \sum_{a}\sum_{b}\sum_{c}C^{abcdef}D_{abcijkl}
    \end{eqnarray}
    Segun este convenio pra contraccion de indices ya no usaremos el simbolo de suma, sino que esto se vera implicito por la repeticion del índice tanto en la parte superior como en la inferior.
\end{itemize}

La notación de indices también nos ayuda a expresar las propiedades de los tensores.
\begin{itemize}
    \item La parte simétrica de un tensor $T_{ab}$ es
    \begin{equation}
        T_{(ab)}=\dfrac{1}{2}\qty(T_{ab}+T_{ba})
    \end{equation}
    \item La parte antisimétrica de un tensor $T_{ab}$ es
    \begin{equation}
        T_{[ab]}=\dfrac{1}{2}\qty(T_{ab}-T_{ba})
    \end{equation}
\end{itemize}

\section{Simetría axial y estacionaria}

Asumamos un espacio axiosimétrico y estacionario. En este espacio la métrica se puede expresar de la siguiente manera:
%
\begin{equation}
\dd{s}^{2} = -V(\rho, z)\qty(\dd{t}-\omega\dd{\phi})^{2}+V(\rho, z)^{-1}\rho^{2}\dd{\phi}^{2}+\Omega(\rho, z)^{2}\qty(\dd{\rho}^{2}+\Lambda(\rho, z)\dd{z}^{2})
\end{equation}
%
Donde $V,\Omega, \Lambda$ son funciones que solo dependen de $\rho$ y $z$la cual en el caso especial en que el espacio es vacío, es decir $R_{ab}=0$ tenemos la siguiente ecuación para $\rho$:
%
\begin{equation}
    D^{a}D_{a}\rho=0 \quad D_{a}: \text{derivada covariantes en la 2D atravesada por } \rho \, z
\end{equation}
%
% \section{Logo Universidad de Concepción}
% \begin{figure}[h]
%     \centering
% \caption{UdeC logo}
%     \includegraphics[width=0.2\columnwidth]{./Images/escudo_unsaac.jpg}
%     \label{fig_logo2}
% \end{figure}

\section{Tensores covariante y contravariante, métrica}\label{sec:tensores}
Un tensor contravariante es del tipo (k,0) y un tensor covariante es del tipo (0,l). Comunmente nos referiremos a los vectores que estan en el espacio tangente $V_{p}$ como contravariantes y a los vectores que estan en el espacio cotangente $V^{*}_{p}$ como covariantes. Es así que, sea el vector $\vec{v}$ podemos descomponerlo en los vectores base de $V_{p}$\footnote{Tambien se puede descomponer en los vectores base de $V^{*}_{p}$.},y $\dim{v_{p}}=3$ \footnote{Para el ejemplo trabajamos en 3 dimensiones, pero la generalización simple.}.

\begin{equation}
    \vec{v}=v^{1}\vu{e}_1+v^{2}\vu{e}_2+v^{3}\vu{e}_3 = v^{i}\vu{e}_i
\end{equation}

y a partir de la proyeccion de este vector hallaremos las componentes covariantes.

\begin{equation}
    v_{j}=\vec{v}\cdot\vu{e}_{j}=v^{i}\vu{e}_i\cdot\vu{e}_{j}
\end{equation}

es así que podemos definir los componentes de la métrica.

\begin{eqnarray}
    g_{ij} = \vu{e}_i\cdot\vu{e}_{j} & \qq{donde} \vu{e}_i,\vu{e}_j \qq{son vectores base de} V_{p}\\
    g^{ij} = \vu{e}^i\cdot\vu{e}^{j} & \qq{donde} \vu{e}^i,\vu{e}^j \qq{son vectores base de} V^{*}_{p}
\end{eqnarray}

Con esta definición de la métrica tenemos algunas propiedades interesantes

\begin{itemize}
    \item Subir y bajar indices
    \begin{eqnarray}
        v_{j} & = v^{i} g_{ij} \\
        v^{j} & = v_{i} g^{ij}
    \end{eqnarray}
    \item Norma de un vector
    \begin{eqnarray}
        \vec{v}\vdot\vec{v}= & v^{i}\vu{e}_i\vdot v^{j}\vu{e}_j=v^{i}v^{j}\vu{e}_i\vdot\vu{e}_{j}=v^{i}v^{j}g_{ij} \\ 
        & \vec{v}\vdot\vec{v} = v^{i} v_{i}
    \end{eqnarray}
\end{itemize}

\section{Problema de Kepler}
Antes de Ley de Gravitación Universal siempre era un problema poder determinar las ecuaciones que rigieran el momvimmiento de los planetas. Sin embargo con la ley de gravitación universal y la segunda Ley de movmiento de Newton se pueden expresar las ecauciones de movimiento.

Pero esta no es la única manera de poder llegar a las ecuaciones de movimiento, otra manera de llegar a ellas es mediante la conservación de la energía. Entonces escribamos la ecuación para la energía de un cuerpo dentro de un potencial gravitacional.

\begin{equation}
    E = \dfrac{1}{2}m\qty(\dot{r}^{2}+r^{2}\sen^{2}(\theta)\dot{\varphi}^{2}+r^{2}\dot{\theta}^{2}) - U(r)
\end{equation}

existe una simetría con respecto a $\theta$ por lo cuál es indiferente el valor que escogamos para esta variable. Por simplicidad escogeremos $\theta=\pi/2$. Entonces la energía es:

\begin{equation}
    E = \dfrac{1}{2}m\qty(\dot{r}^{2}+r^{2}\dot{\varphi}^{2}) + U(r)
\end{equation}

Ahora construyamos el lagrangiano de la partícula.

\begin{equation}
    \mathcal{L} = \dfrac{1}{2}m\qty(\dot{r}^{2}+r^{2}\dot{\varphi}^{2}) - U(r)
\end{equation}

del lagrangiano podemos deducir que $\varphi$ es una coordenada cíclica, entonces si vemos la ecuación de Lagrange correspondiente a la coordenada cíclica podemos obtener
una constante de movimiento que es el momentum asosciado a esa coordenada, \textit{Momentum Angular}.

\begin{equation}
    M = m r^{2} \dot{\varphi}
\end{equation}

entonces reemplazando en la ecuación  de la energía tenemos

\begin{equation}
    E = \dfrac{1}{2}m\dot{r}^{2}+\dfrac{M^{2}}{2mr^{2}} + U(r)
\end{equation}

como podemos ver la ecuacion de movimiento se redujo a una partícula dentro de el nuevo potencial $\dfrac{M^{2}}{2mr^{2}} + U(r)$, a este potencial se le llama \textit{potencial efectivo}. Seguidamente haciendo la regla de la cadena $\dot{r} = \dv{r}{\varphi} \dot{\varphi}$ podemos expresar la energía como

\begin{equation}
    E = \dfrac{1}{2}m\qty[\qty(\dv{r}{\varphi})^{2}\dfrac{M^{2}}{m^{2}r^{4}}+\dfrac{M^{2}}{m^{2}r^{2}}] + U(r)
\end{equation}

despejando $\dv{r}{\varphi}$ tenemos que

\begin{equation}
    \dv{r}{\varphi} = \dfrac{M}{r^{2}}\sqrt{2m\qty[E-U(r)]-\dfrac{M^{2}}{r^{2}}}
\end{equation}

separando variables para poder integrar

\begin{equation}
    \varphi = \int \dfrac{M\dd{r}/r^{2}}{\sqrt{2m\qty[E-U(r)]-M^{2}/r^{2}}} + cte
\end{equation}

reemplazando aqui el potencial gravitacional $U(r) = -\dfrac{\alpha}{r}$ podemos resolver la integral

\begin{equation}
    \varphi = \arccos{\qty(\dfrac{M/r-m\alpha/M}{\sqrt{2mE+\dfrac{m^{2}\alpha^{2}}{M^{2}}}})} + cte
\end{equation}

tomando adecuadamente $\varphi$ tal que la constante sea 0, y además considerando 

\begin{equation}
    p = M^{2}/m\alpha \qc e=\sqrt{1+2EM^{2}/m\alpha^{2}}
\end{equation}

la ecuación de movimiento se puede escribir como

\begin{equation}
    p/r = 1 + e \cos{\varphi}
\end{equation}

Esta es la ecuación de una sección cónica, con unos de sus focos en el origen; además $p$ y $e$ se llaman respectivamente \textit{parámetro} y \textit{excentricidad}.

\section{Vectores de killing}\label{sec:killing}
Se llama vectores de killing a aquellos vectores que cumplen que la derivada de Lie de la métrica con respecto a estos es cero, es decir:

\begin{equation}
    \pounds_{\xi}g_{\mu\nu} = 0
    \label{killingLie}
\end{equation}

esot quiere decir que estos vectores generan un mapa que mantiene invariante a la métrica. Otra forma equivalente de expresar \eqref{killingLie} usando la derivada covariante es:

\begin{equation}
    \nabla_{a}\xi_{b} + \nabla_{b}\xi_{a} = 0
\end{equation}

Cuando tu multiplicas un vector de killing por tangente de una geodésica esta se mantiene constante a lo largo de la geodésica. Esto es

\begin{eqnarray*}
    u^{b}\nabla_{b}(\xi_{a}u^{a}) = & u^{b}u^{a}\nabla_{b}\xi_{a} + \xi_a u^{b}\nabla_{b}u^{a} \\
    = & u^{b}u^{a}\dfrac{1}{2}\qty(\nabla_{a}\xi_{b} + \nabla_{b}\xi_{a})+ \xi_a u^{b}\nabla_{b}u^{a} \\
    = & 0
\end{eqnarray*}

entonces podemos deducir que

\begin{equation}
    \xi_{a}u^{a} = cte
    \label{killingConservation}
\end{equation}

Ahora calculemos los vectores de killing para la métrica de Schwarzschild. La métrica es

\begin{equation}
    \dd{s}^{2} = -c^{2}\qty(1-\dfrac{2G_{N}M}{c^{2}r})\dd{t}^{2} + \qty(1-\dfrac{2G_{N}M}{c^{2}r})^{-1}\dd{r}^{2} + r^{2}\qty(\dd{\theta}^{2}+\sin^{2}{\theta}\dd{\varphi}^{2})
\end{equation}

en donde fijamos $\theta = \pi/2$ y $r_{g} = 2G_{N}M/c^{2}$ obteniendo así

\begin{equation}
    \dd{s}^{2} = -c^{2}\qty(1-\dfrac{r_{g}}{r})\dd{t}^{2} + \qty(1-\dfrac{r_{g}}{r})^{-1}\dd{r}^{2} + r^{2}\dd{\varphi}^{2}
\end{equation}

dada la forma de esta métrica (no hay dependencia temporal y la parte angular es idéntica a Minkowski) podemos deducir los siguientes vectores de killing:

\begin{eqnarray}
    \xi^{ct} = & 1 \\
    \xi^{\varphi} = & 1 
\end{eqnarray}

introduciendo estos vectores en \eqref{killingConservation} obtenemos

\begin{eqnarray}
    g_{ct ct}\xi^{ct}u^{ct} = & -\qty(1-\dfrac{r_{g}}{r})(1)(c\dot{t}) = cte \\
    g_{\varphi \varphi}\xi^{\varphi}u^{\varphi} = & \qty(r^{2})(1)(\dot{\varphi}) = cte 
\end{eqnarray}

para poder determinar estas constantes tenderemos el límite a minkowski, esto es haciendo $r \rightarrow \infty$, para el caso temporal esto es:

\begin{eqnarray*}
    -\qty(1-\dfrac{r_{g}}{r})(c\dot{t}) \Rightarrow -c\dot{t} = -\dfrac{c}{\sqrt{1-\dfrac{v^{2}}{c^{2}}}}
\end{eqnarray*}

desarrollando en series de taylor tenemos 

\begin{equation*}
    g_{ct ct}\xi^{ct}u^{ct} = -(c+\frac{1}{2c}v^{2}) = -\dfrac{E}{mc}
\end{equation*}

donde E es la energía. De la misma manera para $\dot{\varphi}$ podemos encontrar que

\begin{equation*}
    g_{\varphi \varphi}\xi^{\varphi}u^{\varphi} = \dfrac{J}{m}
\end{equation*}

donde $J$ es el momentum angular. Entonces despejando $\dot{t}, \dot{\varphi}$ tenemos

\begin{eqnarray}
    \dot{t} = & \qty(1-\dfrac{r_{g}}{r})^{-1}\dfrac{E}{mc^{2}} \\
    \dot{\varphi} = & \dfrac{J}{mr^{2}}
\end{eqnarray}

\section{Unidades}
Las unidades que usare son las siguientes:

\begin{align}
    \text{tiempo} & = yr\qq{[años]} \\
    \text{distancia} & = AU\qq{[Unidades Astronómicas]} \\
    \text{masa} & = M_{\odot}\qq{[Masas solares]}
\end{align}

en estas nuevas unidades las constantes físicas adoptaran lso siguientes valores

\begin{align}
    G_{N} & \approx 39.409 AU^{3}M_{\odot}^{-1}yr^{-2}, \\
    c & \approx 6.324 \cdot 10^{4} AU yr^{-1}, \\
    \kappa & \approx 6.192 \cdot 10^{-17} AU^{-1}M_{\odot}^{-1}yr^{2}
\end{align}

\section{Runge-Kutta para Schwarzschild}\label{rungeKutta}
Usaré el método de Runge-Kutta de 4 orden para resolver la ecuación

\begin{equation}
    \dv[2]{u}{\varphi}-\frac{3}{2}r_{g}u^{2}+u = \frac{1}{\lambda}
\end{equation}

Runge-Kutta es un método para resolver ecuaciones diferencuiales de primer orden, transformartemos esta ecuación de segundo orden en dos ecuaciones de primer orden relacionadas, esto es:

\begin{align*}
    \dv{u(\varphi)}{\varphi} = & y(\varphi) \\
    \dv{y(\varphi)}{\varphi} = & \frac{1}{\lambda}+\frac{3}{2}r_{g}u^{2}(\varphi)+u(\varphi)
\end{align*}

de este sistema de ecuaciones encontramos las siguientes relaciones de recurrencias

\begin{align}
    y_{n+1} = & y_{n} + \frac{1}{6}(l_{0}+2l_{1}+2l_{2}+l_{3}) \\
    u_{n+1} = & u_{n} + \frac{1}{6}(k_{0}+2k_{1}+2k_{2}+k_{3})
\end{align}

donde

\begin{align}
    l_{0} = & h\qty(\frac{1}{\lambda}-u_{n}+\frac{3}{2}r_{g}u_{n}^{2}) & k_{0} = & hy_{n} \\
    l_{1} = & h\qty[\frac{1}{\lambda}-u_{n}-\dfrac{1}{2}k_{0}+\frac{3}{2}r_{g}\qty(u_{n}+\dfrac{1}{2}k_{0})^{2}] & k_{1} = & h\qty(y_{n}+\frac{1}{2}l_{0}) \\
    l_{2} = & h\qty[\frac{1}{\lambda}-u_{n}-\dfrac{1}{2}k_{1}+\frac{3}{2}r_{g}\qty(u_{n}+\dfrac{1}{2}k_{1})^{2}] & k_{2} = & h\qty(y_{n}+\frac{1}{2}l_{1}) \\
    l_{3} = & h\qty[\frac{1}{\lambda}-u_{n}-k_{2}+\frac{3}{2}r_{g}\qty(u_{n}+k_{2})^{2}] & k_{3} = & h\qty(y_{n}+\frac{1}{2}l_{2})
\end{align}

y para los valores iniciales fijaremos la energía y $u(0)$. Fijando la energía podemos hallar $\dv*{u}{\varphi}$ de \eqref{geodesicas:energia}. Lo que debemos tener cuidado es que nuestro par (posición inicial,energía) debe estar por encima de la curva del potencial efectivo, ya que cualquier punto debajo de esta curva no tiene sentido físico ya que nos daría velocidades imaginarias. 

En el caso de geodesicas $\nl$ es similar solamente que la ecuación diferencial es homogénea, esto quiere decir que $1/\lambda = 0$.

\biblio %Se necesita para referenciar cuando se compilan subarchivos individuales - NO SACAR
\end{document}