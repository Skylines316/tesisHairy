% !TeX root = ../Main.tex
\documentclass[../Main.tex]{subfiles}
\begin{document}
Indicar el cambio anómalo en el momentum angular de los agujeros negros con pelo.
\section{Matriz de Consistencia}

\begin{table}[!ht]
\begin{center}
\begin{tabular}{| m{5em} | m{5em} | m{5em} | m{5em} | m{5em} |}
\hline
Formulación del Problema & Formulación de Objetivos & Formulación de Hipótesis & Justificación & Metodología \\
\hline \hline
\textbf{¿La zona de pelo denso de un agujero negro afecta a sus geodésicas?} & Analizar el efecto de la zona de pelo denso sobre las geodésicas & Esperamos encontrar alguna perturbación por parte de la zona de pelo denso sobre las geodésicas de agujeros negros con pelo & Ya se propusieron antes una conjetura y buscamos entender cual es su impacto sobre esta zona sobre las geodésicas de los agujeros negros con pelo & No Experimental, Explicativa \\
\hline
\end{tabular}
\end{center}
\end{table}

\biblio %Se necesita para referenciar cuando se compilan subarchivos individuales - NO SACAR
\end{document}