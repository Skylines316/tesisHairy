\documentclass[../Main.tex]{subfiles}
\begin{document}

\section{Planteamiento del Problema}
En el estudio de los agujeros negros y teorías cuánticas de campos siempre hubo problemas ya sea por ecuaciones muy complejas, o no lineales; o por que la teoría es muy interactuante y por lo tanto difícil de calcular. Usando la teoría de cuerdas se encontró una relación entre un tipo de teoría Cuántica de Campos y un tipo de geometría relativista. Usando esta dualidad se encontró que se puede calcular la complejidad de un sistema cuántico. Existen antecedentes del cálculo de la complejidad de sistemas cuánticos discretos, y usando esos métodos planteare métodos análogos para poder calcular la complejidad de un sistema cuántico continuo. Y de esta manera aclarar un poco más el concepto de complexity de un sistema continuo.

\section{Formulación del Problema}
\textbf{¿Como hallar la complejidad de un Oscilador Armónico con el formalismo de la matriz de covarianza?}

\section{Objetivo}
Comparar nuestros resultados obtenidos con los métodos de Fubini Study y de Nielsen para obtener la complexity de un Oscilador Armónico Cuántico. Destajar las ventajas del formalismo de la matriz de covarianza para obtener la complejidad de un oscilador armónico cuántico.

\section{Justificación}
Busco mostrar que el formalismo de la matriz de covarianza es mucho más simple para poder realizar los cálculos, los cálculos se hacen mucho más simples en comparación con los métodos de Fubini-Study, y de Nielsen. Además mostrar la posibilidad de utilizar este formalismo para cálculos mucho más complejos.

\section{Limitaciones}
Dado que el formalismo de la matriz de covarianza es relativamente nuevo es difícil encontrar mucha información sobre este.

\section{Delimitaciones}
Existen muchos tipos de campos cuánticos como el campo complejo, el vectorial, etc. Me centrare solamente en el campo escalar tipo Klein Gordon.


%\subsection{Este es un sub título}

%\lipsum[15-17] % dummy text

%\subsubsection{Este es un sub sub título}

%\lipsum[15-17] % dummy text

%\paragraph{Este es un párrafo}

%\lipsum[18-19] % dummy text

%\biblio %Se necesita para referenciar cuando se compilan subarchivos individuales - NO SACAR
\end{document}