% !TeX root = ../Main.tex
\documentclass[../Main.tex]{subfiles}
\begin{document}

\section{Planteamiento del Problema}
Los agujeros negros con pelo fueron extensivamente estudiados en \cite{Anabalon:2013sra,Acena:2013jya}. Las soluciones que usaremos pueden evadir el teorema de no pelo \cite{Hertog:2006rr}, y la estabilidad de los agujeros negros con pelo presentados esta asegurado por su potencial escalar y sus puntos extremos.
Una importante motivación del estudio  de estos agujeros negros con pelo es la construcción de modelos de juguete para agujeros negros super masivos, como Sagitarius A. Esta familia de soluciones tiene un radio de horizonte $r_h$ que es completamente diferente del radio de Schwarzschild $2MG_{N}/c^{2}$, es más, podemos probar que $r_{h}\leq 2MG_{N}/c^{2}$. Entonces existe una región $\mathcal{D}\equiv 2MG_{N}/c^{2}-r_{h}$ llamada zona de pelo denso \cite{}. Esta región podría tener un importante efecto sobre las geodésicas, pierdendo estas su momentum angular y en consecuencia las trayectorias se converterian en un movimiento lineal radial.

\section{Formulación del Problema}
\textbf{¿La zona de pelo denso de un agujero negro afecta a sus geodésicas?}

\section{Objetivo}
Realizar gráficas de las geodésicas de los agujeros negros con pelo para el estudio de su comportamiento en la zona de pelo denso. Seguidamente comparar las geodésicas de agujeros negros con pelo con un agujero negro tipo Schwarzschild. Estas gráficas se realizarán usando métodos numéricos y paquetes de Python.

\section{Justificación}
En \cite{Nunez:1996xv} se propone la conjetura de una zona de pelo denso o hairosphera, entonces estudiar estos agujeros negros ayudará a poder resolver esta conjetura y entender cual es el impacto de esta zona sobre las geodésicas de los agujeros negros con pelo.

\section{Limitaciones}
Dado que las ecuaciones son totalmente no lineales podríamos encontrar dificultades en la gráfica de estas geodésicas. Además, los parámetros a fijar son varios, y al fijar cada uno de estos parámetros se obtiene un agujero negro diferente lo cuál hace practicamente imposible estudiar todos los agujeros negros. 

\section{Delimitaciones}
Son muchos agujeros negros por lo que no podremos estudiarlos todos y nos centraremos en aquellos en los cuales la prescencia de una zona de pelo denso tenga mayor efecto.

%\subsection{Este es un sub título}

%\lipsum[15-17] % dummy text

%\subsubsection{Este es un sub sub título}

%\lipsum[15-17] % dummy text

%\paragraph{Este es un párrafo}

%\lipsum[18-19] % dummy text

%\biblio %Se necesita para referenciar cuando se compilan subarchivos individuales - NO SACAR
\end{document}